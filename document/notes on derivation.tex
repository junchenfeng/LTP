\documentclass{article}

\usepackage{amsmath}
\usepackage{color}
\usepackage{ amssymb }



\usepackage{float}


\title{Notes on the Derivation}
\author{Junchen Feng}

\begin{document}
\maketitle

\section{Derivation on the EM Algorithm}
All notation follows Streeter(2015)'s paper.\\
The log likelihood function for the observed data is \[
\mathcal{L} = \sum_{s}\{log[\sum_{j}p_j\prod_{t}\mathcal{B}(q_t^j, v_t)]\}
\]
Also borrow the notation
\begin{align*}
L_{s,j} &= p_j\prod_{t}\mathcal{B}(q_t^j, v_t) \\
z_{s,j} &= \frac{L_{s,j}}{\sum_{j'} L_{s,j'}}
\end{align*}


\subsection{$q_t^j$}
\begin{align*}
\frac{\partial \mathcal{L} }{\partial q_t^j} &=  \sum_{s} \frac{\frac{\partial L_{s,j}}{\partial q_t^j}}{\sum_{j}L_{s,j}}\\
&= \sum_{s} (\frac{\partial L_{s,j}}{\partial q_t^j}\frac{1}{L_{s,j}})\frac{L_{s,j}}{\sum_{j}L_{s,j}}\\
&= \sum_{s} \frac{\partial log(L_{s,j})}{\partial q_t^j} z_{s,j}\\
&= \frac{1}{q_t^j(1-q_t^j)}\sum_{s} (v_t^s(1-q_t^j)-(1-v_t^s)q_t^j)z_{s,j}\\
&= \frac{1}{q_t^j(1-q_t^j)}\sum_{s} (v_t^s-q_t^j)z_{s,j}\\
&=0
\end{align*}
From here it is easy to derive that
\[
\hat{q}_{t+1}^j = \frac{\sum_{s}v_t^sz_{s,j}}{\sum_{s}z_{s,j}}
\]

Streeter then use beta prior to shrink the estimate by
\[
q_{t+1}^j = \frac{\alpha-1+\sum_{s}v_t^sz_{s,j}}{\alpha+\beta-2+\sum_{s}z_{s,j}}
\]
Note that the original $v_s^j$ is likely a typo.

\subsection{$p_j$}
To add the constraint that $\sum_{j}p_j = 1$, use the lagrange multipler so that
\begin{align*}
\frac{\partial \mathcal{G} }{\partial p^j} &=  -\lambda + \sum_{s} \frac{1}{p^j}\frac{L_{s,j}}{\sum_{j}L_{s,j}} = -\lambda + \frac{1}{p^j} \sum_{s} z_{s,j} = 0\\
\frac{\partial \mathcal{G} }{\partial \lambda} &= 1 - \sum_{j}p^j = 0
\end{align*}

Some simple algebra yields
\[
p^j_{t+1} = \frac{\sum_{s} z_{s,j}}{\sum_{j'}\sum_{s} z_{s,j'}}
\]

\section{Derivation on the Equivalence to BKT}
The weight can be derived from the fact that the probability of mastering the knowledge component has to be the same at time $t$ has to be the same.
\[
P(M_t=1) = \sum_j p_jI(q_t^j=0.8)
\]

Let j = 1,2,3 denotes the solid line, dash line and dot line respectively.\\

In the first period
\[P(M_1=1)=p_1 = 0.5\]

In the second period
\begin{align*}
P(M_2=1) &=p_1+p_2\\
P(M_2=1) &= P(M_2=1,M_1=1)+P(M_2=1,M_1=0)\\
		 &= P(M_2=1|M_1=1)P(M_1=1)+P(M_2=1|M_1=0)P(M_1=0)\\
		 & = 1*0.5+0.5*0.5\\
p_2 &= 0.25
\end{align*}

It can be proved by generalization that
\[
p_j = 0.5^j
\]

\end{document}

